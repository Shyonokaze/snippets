\subsection{Data reduction \& analysis}

Data analysis was performed using two custom scripts written
for the Open Source \code{R} statistical and data analysis language.
This approach assures maximum reproducibility.

The first script, \code{01-cleanData.R}, retrieves the desired
area percent data from the .csv file, checks the entries, and
stores the R object for later processing (this approach is our
standard, and is most useful for larger projects where data
cleaning often involves significant initial data processing.)

The second script \code{02-anaArea.R} does the remainder of
the data analysis, writing the plots and tables for this report.
The script computes the usual descriptive statistics, constructs
a panel plot with the histogram of area percents from the 60
fields, with the empirical kernel density function and median
value superimposed. The second plot in the panel is a boxplot
with a visual description of the robust statistics and the
outliers plotted. The final plot in the panel is a
quantile-quantile plot of the experimental quantiles as a
function of the quantiles from a standard normal distribution.
Deviations from linearity suggest non-normality of sample
distribution.

Finally, the script tests the data for non-normality using a
robust Jarque-Bera test (from the \code{lawstat} R-package) and
writes tables for the report (using the \code{xtable} R-package).
The report was written using \LaTeX .
The entire process of data analysis and report generation may
be reproduced using a single click of a command script
\code{rule-them-all.cmd}.


\endinput
