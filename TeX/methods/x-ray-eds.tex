\subsection{X-Ray Energy Dispersive Spectroscopy}

X-Ray Energy Dispersive Spectra (EDS) were recorded from selected
regions of the specimen using an EDAX Genesis system with an RTEM
ultra-thin window SiLi detector with a 0.13 sr solid angle. The
specimen was mounted in a ``low-background'' specimen holder,
where the material around the specimen is constructed of Be to
minimize the production of spurious x-rays.

Spectra were recorded with electron irradiation at 200 kV with the
microscope operated in ``nanoprobe'' mode with a 50 \textmu m diameter
thick Pt ``top hat'' condenser lens aperture to minimize the
transmission of ``hard x-rays'' (high-energy) down the column.

Despite these precautions, the collimation of the detector
is not sufficient to eliminate the detection of small amounts
of Fe and Co (produced by backscattered electrons hitting the
pole piece of the objective lens and producing undesired x-rays.
A small Cu peak is also observed in most spectra. This is produced
by fluorescence by backscattered electrons hitting the support grid.

In this case, the specimen holder was tilted 7.5\degree\ toward
the detector to maximize detection efficiency. The first condenser
lens was set to ``spot size 3'' and the illumination adjusted to
form a demagnified image of the condenser aperture on the specimen.
This produced an evenly-illuminated disk 1.3 \textmu m in diameter,
averaging over a convenient size region of the specimen and providing
approximately 900 x-ray cps on the detector. Spectra were recorded 
three representative areas of each of the regions near the bottom,
middle, and top surfaces of the electrode cross-section.

The X-ray peaks were identified using the EDAX Genesis software
and the ``Auto'' processing feature was configured to automatically
process each spectrum, subtracting the continuum intensity and
computing the peak integrals.

An overlay of a spectrum from one such area near the
top of the electrode and a background spectrum is shown
in Figure~\ref{fig:edsSpectra}. Note that this is displayed on
a logarithmic scale and shows that the signal/noise ratio of the
spectra from the areas is good.

Because we do not have standards available, a full quantitative
analysis is not feasible. However, the ratio of peak areas is
proportional to the ratio of concentrations, so the semi-quantitative
analysis described below was feasible.

The EDAX ``Auto'' spectrum processing routine writes the peak
integrals to a comma-delimited text file (.csv) that is easily
processed by other analysis software. We processed the file using
a script for the Open-Source, cross-platform, R Statistical
Programming Language\footnote{We use the current (version 2.15.1),
available from any  \href{http://cran.r-project.org/}{CRAN mirror}.}.

A custom R Script, ``01-anaEDS.R'', reads in the data and creates
dataframes containing the replicate measurements from the background,
the bottom, middle, and the top regions of the electrode. The script
computes the average background peak integral for each of the 
x-ray peaks measured ( C-K, O-K, F-K, S-K, Ca-K, Cu-K, and Pt-L).
The script then subtracts this average background intensity from
every other spectrum in the data set. Next, the script normalizes
the intensity in each spectrum to the C K line in that spectrum,
computing the relative concentration of the element with respect
to C. These results are stored as a data frame for each region
of the electrode (top, middle, bottom). The mean element/C ratio
and the standard error are then computed and stored as a datafame.

The script next attempts to compute the catlyst/ionomer ratio for
each region along with the standard error. The script computes this
value, first using the Pt-L/S-K ratio and then using the Pt-L/F-K
ratio. The script computes the standard error assuming that the
covariance term is negligible and sums the component standard errors
in quadrature.

The script uses the ``xtable'' package to write out tables for
inclusion in this report (composed in \LaTeX).

\endinput
